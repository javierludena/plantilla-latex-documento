\documentclass{tudelf-report}

% --- Metadatos del Documento ---
% Define el título, subtítulo, autor y fecha para tu informe.
% Esta información es utilizada por la portada.
\title{Consilium Functionalis}
\subtitle{Altia Control Tower}
\author{Epiteto}
\supervisor{Iago Rodríguez-Quintana}
\distribution{Javier Ludeña}
\date{\today}

% --- Variable para el cliente ---
\newcommand{\clientname}{Epiteto}

% --- Variables para Control de Versiones ---
% Define la versión del documento y responsables
\newcommand{\docversion}{1.0}
\newcommand{\docauthor}{Séneca}
\newcommand{\docvalidator}{Marco Aurelio}

\begin{document}

% --- Configuración numeración ---
\setcounter{chapter}{1}

% --- Portada ---
% El comando \maketitle genera la portada personalizada.
\maketitle

\newpage
		\thispagestyle{fancy}

		\vspace*{2cm}

		% Título de la página
		\begin{center}
			{\Large\textbf{\sffamily\color{dark-blue}CONTROL DE VERSIONES}}
		\end{center}

		\vspace{1cm}

		% Definición de un nuevo estilo para las cajas de la tabla de versiones
		\newtcolorbox{versionbox}[2][]{
			colback=light-grey,
			colframe=light-grey,
			boxrule=0pt,
			arc=0mm,
			boxsep=5pt,
			left=15pt,
			title=#2,
			coltitle=dark-blue,
			fonttitle=\bfseries\large,
			attach boxed title to top left={xshift=15mm, yshift=-3.5mm},
			boxed title style={empty},
			#1,
			underlay={
				\begin{tcbclipinterior}
					\fill[light-blue-original] (frame.north west) rectangle ([xshift=8pt]frame.south west);
				\end{tcbclipinterior}
			}
		}

		\begin{versionbox}{VERSIÓN \docversion}
			\begin{tabularx}{\textwidth}{p{0.25\textwidth}X}
				\textbf{VERSIÓN} & \docversion \\ 
			\end{tabularx}
		\end{versionbox}

		\vspace{1cm}

		\begin{versionbox}{RESPONSABLES}
			\begin{tabularx}{\textwidth}{p{0.25\textwidth}X}
				\textbf{ELABORACIÓN} & \docauthor \\ 
				\textbf{VALIDACIÓN} & \docvalidator \\ 
			\end{tabularx}
		\end{versionbox}

		\vspace{1cm}


\newpage % Mueve el contenido a la página siguiente de la portada

% --- Tabla de Contenidos ---
\tableofcontents
\newpage

% --- Contenido del Informe ---
% Incluye las secciones desde archivos separados
\input{sections/01-objeto.tex}
\chapter{Planteamiento de los requisitos}

Lorem ipsum dolor sit amet, consectetur adipiscing elit, sed do eiusmod tempor incididunt ut labore et dolore magna aliqua. Quis ipsum suspendisse ultrices gravida. Risus commodo viverra maecenas accumsan lacus vel facilisis.

\section{Requisitos Funcionales}

Ut enim ad minim veniam, quis nostrud exercitation ullamco laboris nisi ut aliquip ex ea commodo consequat. Duis aute irure dolor in reprehenderit in voluptate velit esse cillum dolore eu fugiat nulla pariatur.

\subsection{Lorem ipsum dolor sit}
Sed ut perspiciatis unde omnis iste natus error sit voluptatem accusantium doloremque laudantium, totam rem aperiam, eaque ipsa quae ab illo inventore veritatis et quasi architecto beatae vitae dicta sunt explicabo. Nemo enim ipsam voluptatem quia voluptas sit aspernatur aut odit aut fugit.

\subsection{Consectetur adipiscing elit}
At vero eos et accusamus et iusto odio dignissimos ducimus qui blanditiis praesentium voluptatum deleniti atque corrupti quos dolores et quas molestias excepturi sint occaecati cupiditate non provident.

\subsection{Sed do eiusmod tempor}
Similique sunt in culpa qui officia deserunt mollitia animi, id est laborum et dolorum fuga. Et harum quidem rerum facilis est et expedita distinctio. Nam libero tempore, cum soluta nobis est eligendi optio cumque nihil impedit quo minus.

\subsection{Incididunt ut labore}
Id quod maxime placeat facere possimus, omnis voluptas assumenda est, omnis dolor repellendus. Temporibus autem quibusdam et aut officiis debitis aut rerum necessitatibus saepe eveniet ut et voluptates repudiandae sint et molestiae non recusandae.

\section{Requisitos No Funcionales}

Itaque earum rerum hic tenetur a sapiente delectus, ut aut reiciendis voluptatibus maiores alias consequatur aut perferendis doloribus asperiores repellat.

\subsection{Et dolore magna aliqua}
Ut enim ad minima veniam, quis nostrum exercitationem ullam corporis suscipit laboriosam, nisi ut aliquid ex ea commodi consequatur. Quis autem vel eum iure reprehenderit qui in ea voluptate velit esse quam nihil molestiae consequatur.

\subsection{Vel illum qui dolorem}
Eum fugiat quo voluptas nulla pariatur. At vero eos et accusamus et iusto odio dignissimos ducimus qui blanditiis praesentium voluptatum deleniti atque corrupti quos dolores et quas molestias excepturi.

\subsection{Sint occaecati cupiditate}
Non provident, similique sunt in culpa qui officia deserunt mollitia animi, id est laborum et dolorum fuga. Et harum quidem rerum facilis est et expedita distinctio.
\chapter{Arquitectura Técnica}

Nam libero tempore, cum soluta nobis est eligendi optio cumque nihil impedit quo minus id quod maxime placeat facere possimus, omnis voluptas assumenda est, omnis dolor repellendus.

\section{Componentes del Sistema}

Temporibus autem quibusdam et aut officiis debitis aut rerum necessitatibus saepe eveniet ut et voluptates repudiandae sint et molestiae non recusandae.

\subsection{Itaque earum rerum}
Hic tenetur a sapiente delectus, ut aut reiciendis voluptatibus maiores alias consequatur aut perferendis doloribus asperiores repellat. Sed ut perspiciatis unde omnis iste natus error sit voluptatem.

\subsection{Accusantium doloremque}
Laudantium, totam rem aperiam, eaque ipsa quae ab illo inventore veritatis et quasi architecto beatae vitae dicta sunt explicabo. Nemo enim ipsam voluptatem quia voluptas sit aspernatur aut odit aut fugit.

\subsection{Sed quia consequuntur}
Magni dolores eos qui ratione voluptatem sequi nesciunt. Neque porro quisquam est, qui dolorem ipsum quia dolor sit amet, consectetur, adipisci velit, sed quia non numquam eius modi tempora incidunt.
\chapter{Plan de Implementación}

Ut labore et dolore magnam aliquam quaerat voluptatem. Ut enim ad minima veniam, quis nostrum exercitationem ullam corporis suscipit laboriosam, nisi ut aliquid ex ea commodi consequatur.

\section{Fases del Proyecto}

Quis autem vel eum iure reprehenderit qui in ea voluptate velit esse quam nihil molestiae consequatur, vel illum qui dolorem eum fugiat quo voluptas nulla pariatur.

\subsection{At vero eos et accusamus}
Et iusto odio dignissimos ducimus qui blanditiis praesentium voluptatum deleniti atque corrupti quos dolores et quas molestias excepturi sint occaecati cupiditate non provident.

\subsection{Similique sunt in culpa}
Qui officia deserunt mollitia animi, id est laborum et dolorum fuga. Et harum quidem rerum facilis est et expedita distinctio. Nam libero tempore, cum soluta nobis est eligendi optio.

\subsection{Cumque nihil impedit}
Quo minus id quod maxime placeat facere possimus, omnis voluptas assumenda est, omnis dolor repellendus. Temporibus autem quibusdam et aut officiis debitis aut rerum necessitatibus saepe eveniet.

\subsection{Ut et voluptates}
Repudiandae sint et molestiae non recusandae. Itaque earum rerum hic tenetur a sapiente delectus, ut aut reiciendis voluptatibus maiores alias consequatur aut perferendis doloribus asperiores repellat.

\end{document}